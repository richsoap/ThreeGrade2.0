\documentclass[UTF8]{article}
\usepackage{CJK}
\usepackage{cite}
\usepackage{multirow}
\begin{document}
\begin{CJK}{UTF8}{gkai}
\title{电磁波理论与应用导论课后作业}
\author{杨庆龙\quad 1500012956}
%\institute{yangqinglong@pku.edu.cn}
\maketitle

\part{第一部分\quad 课后问题回答}
\section{电磁波理论基础}
\subsection{狭义相对论的基本假设是什么?}
\subsubsection{相对性原理}
一切物理定律,在所有惯性系中均有效,即一切物理定律的方程式在洛仑兹变换下都具有形式不变性。
\subsubsection{光速不变性}
光在真空中传播的速度恒定为光速,且与光源的运动状态无关,与光的传播方向无关,与观察者所处惯性系状态无关。
\section{电磁频谱管理与应用}
\subsection{手机通信使用了哪些频率,未来5G使用什么频率?}
工信部公布的5G通信网络使用3000MHz-3600MHz和4800MHz-5000MHz两个频段。
中国有三大移动通信运营商,而且每个运营商都支持2G,3G,4G三代网络,又加上同一代通信网络也会有不同的解决方案,所以中国国内手机通信的频率使用非常复杂。详情见以下表格:

\begin{table}[!htbp]
  \centering
\caption{中国移动频率使用表}
\begin{tabular}{|c|c|c|c|}
\hline
通信标准&通信技术&上行频率/MHz&下行频率/MHz\\
\hline
\multirow{2}*{2G}&GSM800&885-909&930-954\\
\cline{2-4}
GSM1800&1710-1725&1805-1820\\
\cline{2-4}
3G&TD-SCDMA&2010-2025&2015-2025\\
\cline{1-4}
\multirow{3}*{4G}&\multirow{3}*{LD-LTE}&1880-1890&1880-1890\\
\cline{2-4}
2320-2370&2320-2370\\
\cline{2-4}
2575-2635&2575-2635\\
\hline
\end{tabular}
\begin{tabular}{|c|c|c|c|c|}
\multirow{6}*{中国联通}&\multirow{2}*{2G}&GSM800&909-915&954-960\\
\cline{3-5}
GSM1800&1745-1755&1840-1850\\
\cline{3-5}
3G&WCDMA&1940-1955&2130-2145\\
\cline{2-5}
\multirow{3}*{4G}&\multirow{2}*{TD-LTE}&2300-2320&2300-2320\\
\cline{4-5}
2555-2575&2555-2575\\
\cline{4-5}
FDD-LTE&1755-1765&1850-1860\\
\cline{3-5}
\multirow{5}*{中国电信}&2G&909-915&954-960\\
\cline{2-5}
3G&CDMA2000&1920-1935&2110-2125\\
\cline{2-5}
\multirow{3}*{4G}&\multirow{2}*{TD-LTE}&2370-2390&2370-2390\\
\cline{4-5}
2635-2655&2635-2655\\
\cline{4-5}
FDD-LTE&1765-1780&1860-1875\\
\hline
\end{tabular}
\end{table}

\section{电磁波的发射与传播}
\subsection{什么是地波,天波,视距传播和超视距传播?}
\subsubsection{地波}
沿陆地/海洋表面传播的电磁波,距离几百千米至几千千米,主要使用长波。
\subsubsection{天波}
天波是指利用电离层反射或折射回地球进行传播的电磁波,距离可超过1万千米,主要使用短波频段。由于电离层容易受太阳活动影响,所以天波信号不够稳定。
\subsubsection{视距传播}
视距传播使用超短波,微波作地面通信和广播,其传播距离与地面上人的视线距离相仿,一般不超过50km。
\subsubsection{超视距传播}
电磁波除了利用电磁波反射实现超视距传播,还可以利用低层大气的分层特征通过反射和折射实现超视距传输。
\section{电磁波与媒质和物体的作用}
\subsection{请解释电磁波/光波的极化}
电磁波的极化即为电场矢量末端随时间变化的特性。右手拇指指向传播方向,如果电场矢量旋转方向与四指弯曲方向一致,称为右手极化;相反则称为左手极化;若电场矢量方向不旋转则为线极化。可分为:
\begin{itemize}
  \item 线极化:电场矢量方向不变,但大小可变;
  \item 圆极化:电场矢量大小不变,但方向随时间旋转;
  \item 椭圆极化:电场矢量的大小和方向都随时间变化。
\end{itemize}

\section{电磁波作为信息的载体-信号调制}
\subsection{通信信号调制一般有哪些?}
\subsubsection{模拟信号调制}
\begin{itemize}
\item 幅度调制(AM)
\item 频率调制(FM)
\item  角度调制
\end{itemize}
\subsubsection{数字信号调制}
\begin{itemize}
\item  二进制幅度键控(BASK)
\item  二进制频移键控(BFSK)
\item  四进制频度键控(QFSK)
\item  二进制相移键控(BPSK)
\item  四进制相移键控(QPSK)
\item  正交调制(QAM)
\end{itemize}
\clearpage
\part{第二部分 电磁频谱与应用综述}

Ministryofinformationindustrypeople'srepublicofChina. 中华人民共和国无线电频率划分规定[M]. 人民邮电出版社, 2003.

电磁波看不见摸不着,但又无处不在。但不同电磁波却有着完全不同的性质,微波可以被用来加热食物,X光可以用来机场安检,红外更是可以帮助我们在黑暗中看清物体。只有充分了解不同波长电磁波的性质,我们才能够利用电磁波制造出产品,改善人类的生活。下文就将按照低频到高频的顺序,介绍不同波长电磁波的相关性质和应用。
\setcounter{section}{0}
\section{极低频(ELF)}
极低频电磁波指的是频率为3Hz到30Hz,波长为10000公里到100000公里电磁波。这种电磁波波长极长,远远大于普通地表起伏的物理尺度。因此,这种电磁波可以轻松绕过地表障碍物,实现地表远距通信。此外,极低频电磁波还可以被电离层反射,这就使得地表和电离层构成了一个巨型波导,使得该电磁波甚至有能力实现全球通信。此外,由于很少有分子的振动频率低至几十Hz,所以这种电磁波在空气和水中的衰减都很小[Jursa, Adolph S., Ed. (1985). Handbook of Geophysics and the Space Environment, 4th Ed (PDF). Air Force Geophysics Laboratory, U.S. Air Force. pp. 10.25–10.27.],因此该电磁波还曾被用于潜艇通信[Barr R, Jones D L, Rodger C J. ELF and VLF radio waves [J]. Journal of Atmospheric and Solar-Terrestrial Physics, 2000, 62(17):1689-1718.],但由于其带宽实在有限,所以能传递的信息很少,现在已经不再使用。
该电磁波由于带宽太窄,所以多少应用方面的研究,但其传播距离很广的特点却引起了不少健康方面的关注。
%http://www.wanfangdata.com.cn/details/detail.do?_type=perio&id=bqeykdxxb201202075#
%http://www.wanfangdata.com.cn/details/detail.do?_type=perio&id=zhldwszyb201308014
%http://www.wanfangdata.com.cn/details/detail.do?_type=perio&id=zgyfyxzz200805019
\section*{超低频(SLF)}
超低频指的是频率介于30Hz到300Hz间的电磁波,常见电力传输所使用的即为该波段的电磁波,这也是该电磁波最常见的用途。与ELF类似的,该波段电磁波在水中传播的衰减也很微弱,因此也被用于潜艇通信[ "Navy gets new facility to communicate with nuclear submarines prowling underwater". The Times of India. 31 July 2014.]
\section{特低频(ULF)}
特低频指的是300Hz到3kHz的电磁波。这个频段在地球物理学科十分有用,因为不论是地震还是等离子层扰动都会产生特低频电磁波,也就可以通过检测这些电磁波实现对地球物理状态的监控。
此外,该波段的电磁波还具有穿透地表的能力,因此,该波段电磁波还被北约用于早期的地面通信,以实现山地通信。由于其带宽比较有限,所以现在已经不再军用,但依然被广泛采用与挖矿等特殊工作场所。
\section{甚低频(VLF)}
甚低频指的是3kHz到30kHz的电磁波。与ELF波类似,该波段的电磁波一样可以绕过山地,被电离层反射,进而实现远距传输。但是,与ELF不同,VLF主要是在电离层界面上传播,也因此更容易受太阳活动影响,远距传输时噪声也比ELF波大很多。
由于其具有一定带宽而又可以传播很远的距离,除了被用于最基本的声音信号通信外,也常用于导航和授时。此外,该波段也被用于潜艇通信,但通信距离不如SLF那么深,对发射机的功率也有一定的要求,但其使用频移键控调制方式时,传输速率最高能到75bit/s,已具有传递简单战术命令的命令的能力,和ELF只能传递"入/出海底"的指令相比又进了一大步。
\section{低频(LF)}
低频电磁波频率为30kHz到300kHz。由于这个波段的电磁波衰减比较慢,而带宽又足够宽,能够调制具有一定带宽的语音信号,故最常用于AM无线电广播。与VLF类似,该波段也被用于授时系统和导航系统。
因为只需要一般的晶振就能产生相应波段的基频信号,而调制也不涉及过于复杂的数学物理过程,所以该波段也是无线电爱者最常用的波段.
\section{中频(MF)}
中频电磁波位于300kHz到3MHz的范围内。与LF,VLF类似,该波段也用于授时,导航与广播。该波段的电磁波能够传递一定的距离但又不会过远,发射设备和接收设备都可以做到轻量化和低能耗,所以非常适合用于海上船只间的通信和陆地与海上船只间的通信。
该波段电磁波一样会被电离层反射,实现超视距通信,但通信状态极其容易受电离层状态影响。当电磁波被波动较大的D层反射时,就会带来大量噪声,超视距通信效果将大打折扣;而到了夜晚,随着太阳活动对相应区域电离层的影响减弱,该电磁波就能被较为稳定的F层反射,并实现效果非常优秀的超视距通信。
\section{高频(HF)}
高频为3MHz到30MHz的范围内。该波段主要用于需要较大带宽和具有一定距离的通信场景,常见的有军用短波通信系统,地空通信系统,短波区域广播,岸舰通信系统,超视距雷达,全球海上遇险和安全系统。
\section{甚高频(VHF)}
甚高频指的是频率位于30MHz到300MHz的无线电波。与高频信号类似,甚高频也被广泛应用于航空通信,航海通信以及业余无线电通信。
由于甚高频电磁波能携带比高频电磁波多很多的信息,所以除了用于声音广播外,还被用于电视广播。但是,甚高频电磁波波长较短,所以很容易因为较大障碍物的遮挡而不能很好地传输。此外,该频段电磁波不能被电离层反射,所以一般情况下该波段电磁波的直线传播距离为160km左右。
由于该波段电磁波波长较短,天线尺寸已经小到可以由单人携带,基于该特点设计出了无线对讲机等个人通信设备。
\section{特高频(UHF)}
特高频指的是300MHz到3GHz的电磁波,波长已经小于1m,所以天线尺寸能做得比较小,适合作为移动通信频段。GPS,Wi-Fi,蓝牙等使用的都是该频段电磁波。
但也因为无线电波波长很短,所以很容易被较大的障碍物遮挡,然而该电磁波具有穿透墙壁的能力,所以并不妨碍该电磁波作为室内通信波段。该电磁波在空气中衰减得很快,即使在没有遮挡的环境下也只能传播60km左右,若再考虑上城市环境下的的建筑物遮挡,该电磁波的传播距离会更加有限,所以非常适合用于组建蜂窝移动通信网络。
该电磁波可以沿着对流层传播,所以可以借助该特性实现较远距离的特高频传输,但对流层气象状况非常不稳定,所以传播的有效距离也很不稳定。
\section{超高频(SHF)}
超高频指的是3GHz到30GHz的电磁波,波长处于厘米量级,正因如此,该波段的天线能够做得非常小,常被用于点对点通信系统,数据链和雷达系统。
该频率电磁波波长太短,与较低频率的电磁波不同,该频率电磁波会被地面和电离层反射,而非沿着介质界面传播。因此可以将该电磁波波束朝向电离层发射,然后反射到接收天线上,实现远距通信,弥补其无法绕过大型障碍物的缺点。由于其只是在电离层反射而非沿着电离层传播,所以电离层状态对通信质量的影响相对于其他波长会好一些。
该波段电磁波指向性非常好,且能够在大部分的金属表面有着不错的反射系数,所以被用于制造X波段雷达,用于探测飞机,舰船,甚至潜艇。
\section{极高频(EHF)}
极高频电磁波频率为30GHz到300GHz,波长为毫米量级,具有非常好的指向性,但由于其与大气中大部分分子的振荡频率相近,所以传播距离非常有限,不能实现大气层内远距通信。但也正由于其可以与大气分子相互作用,所以极高频电磁波被用于制造气象雷达,通过检测雷达反射波实现对大气状态的检测。
虽然该频段电磁波传播距离很短,但其带宽很宽,能够很容易地实现高速率信息传递,常被用于无线动态数据链架设。
由于该频段电磁波能够和水分子作用,提高水温,因此该电磁波也被用于制造非致命性微波武器,让被照射到的人感到无法忍受的痛苦却不会有生命危险。
\section{太赫兹(THF)}
太赫兹为300GHz到3THz的电磁波,该频段电磁波位于光学和无线电的交界处,所以拥有很多特殊的性质。
\end{CJK}
\end{document}
