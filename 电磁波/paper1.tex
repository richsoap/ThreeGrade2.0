\documentclass[UTF8]{ctexart}
\usepackage{CJK}
\usepackage{tex_utils}
\usepackage{amsmath, amssymb}
\usepackage{color}
\usepackage{cite}
\begin{document}
\title{电磁波理论与应用导论课后作业}
\author{杨庆龙\quad 1500012956}
\institute{yangqinglong@pku.edu.cn}
\maketitle

\part{第一部分\quad 课后问题回答}
\section{电磁波理论基础}
\subsection{狭义相对论的基本假设是什么?}
\subsubsection{相对性原理}
一切物理定律,在所有惯性系中均有效,即一切物理定律的方程式在洛仑兹变换下都具有形式不变性。
\subsubsection{光速不变性}
光在真空中传播的速度恒定为光速,且与光源的运动状态无关,与光的传播方向无关,与观察者所处惯性系状态无关。
\section{电磁频谱管理与应用}
\subsection{手机通信使用了哪些频率,未来5G使用什么频率?}
中国有三大移动通信运营商,而且每个运营商都支持2G,3G,4G三代网络,又加上同一代通信网络也会有不同的解决方案,所以中国国内手机通信的频率使用非常复杂。详情见以下表格:

\begin{table}[!htbp]
\caption{中国移动频率使用表}
\begin{tabular}{|c|c|c|c|c|}
\hline
服务提供商&通信标准&通信技术&上行频率/MHz&下行频率/MHz\\
\hline
\multirow{6}*{中国移动}&\multirow{2}*{2G}&GSM800&885-909&930-954\\
\cline{3-5}
GSM1800&1710-1725&1805-1820\\
\cline{3-5}
3G&TD-SCDMA&2010-2025&2015-2025\\
\cline{2-5}
\multirow{3}*{4G}&\multirow{3}*{LD-LTE}&1880-1890&1880-1890\\
\cline{3-5}
2320-2370&2320-2370\\
\cline{3-5}
2575-2635&2575-2635\\
\multirow{6}*{中国联通}&\multirow{2}*{2G}&GSM800&909-915&954-960\\
\cline{3-5}
GSM1800&1745-1755&1840-1850\\
\cline{3-5}
3G&WCDMA&1940-1955&2130-2145\\
\cline{2-5}
\multirow{3}*{4G}&\multirow{2}*{TD-LTE}&2300-2320&2300-2320\\
\cline{4-5}
2555-2575&2555-2575\\
\cline{4-5}
FDD-LTE&1755-1765&1850-1860\\
\cline{3-5}
\multirow{5}*{中国电信}&2G&909-915&954-960\\
\cline{2-5}
3G&CDMA2000&1920-1935&2110-2125\\
\cline{2-5}
\multirow{3}*{4G}&\multirow{2}*{TD-LTE}&2370-2390&2370-2390\\
\cline{4-5}
2635-2655&2635-2655\\
\cline{4-5}
FDD-LTE&1765-1780&1860-1875\\
\hline
\end{tabular}
\end{table}

\section{电磁波的发射与传播}
\subsection{什么是地波,天波,视距传播和超视距传播?}
\subsubsection{地波}
地波是指具有沿不同介质界面传播的性质的电磁波,长波即为其中之一。
\subsubsection{天波}
天波是指利用电离层反射或折射回地球进行传播的电磁波,主要使用短波频段。
\subsubsection{视距传播}
视距传播使用超短波,微波作地面通信和广播,其传播距离与地面上人的视线距离相仿,一般不超过50km。
\subsubsection{超视距传播}
使用短波会受电离层影响的性质,实现不受地球曲率的限制的传播。
\section{电磁波与媒质和物体的作用}
\subsection{请解释电磁波/光波的极化}
电磁波是横波,其电场矢量与传播方向垂直,当电磁波穿过具有极性的介质时,电场就会与介质作用









\end{document}


中国移动
|通信技术|上行频率/MHz|下行频率/MHz|
|--|--|--|
|GSM800(2G)|885-909|930-954|
|GSM1800(2G)|1710-1725|1805-1820|
|TD-SCDMA(3G)|2010-2025|2010-2025|
|LD-LTE(4G)|1880-1890//2320-2370//2575-2635|1880-1890//2320-2370//2575-2635|

#### 中国联通
|通信技术|上行频率/MHz|下行频率/MHz|
|--|--|--|
|GSM800(2G)|909-915|954-960|
|GSM1800(2G)|1745-1755|1840-1850|
|WCDMA(3G)|1940-1955|2130-2145|
|TD-LTE(4G)|2300-2320//2555-2575|2300-2320//2555-2575|
|FDD-LTE(4G)|1755-1765|1850-1860|

#### 中国电信
|通信技术|上行频率/MHz|下行频率/MHz|
|--|--|--|
|CDMA(2G)|825-840|870-885|
|CDMA2000(3G)|1920-1935|2110-2125|
|TD-LTE(4G)|2370-2390//2635-2655|2370-2390//2635-2655|
|FDD-LTE(4G)|1765-1780|1860-1875|
