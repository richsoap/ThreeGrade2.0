\documentclass[UTF8]{article}
\usepackage{CJK}
\usepackage{cite}
\usepackage{multirow}
\begin{document}
\begin{CJK}{UTF8}{gkai}
\title{电磁波理论与应用导论课后作业}
\author{杨庆龙\quad 1500012956\\yangqinglong@pku.edu.cn}
\date{2018.4}
\maketitle

\part{课后问题回答}
\section{电磁波理论基础}
\subsection{狭义相对论的基本假设是什么?}
\subsubsection{相对性原理}
一切物理定律,在所有惯性系中均有效,即一切物理定律的方程式在洛仑兹变换下都具有形式不变性。
\subsubsection{光速不变性}
光在真空中传播的速度恒定为光速,且与光源的运动状态无关,与光的传播方向无关,与观察者所处惯性系状态无关。
\section{电磁频谱管理与应用}
\subsection{手机通信使用了哪些频率,未来5G使用什么频率?}
工信部公布的5G通信网络使用3000MHz-3600MHz和4800MHz-5000MHz两个频段。
中国有三大移动通信运营商,而且每个运营商都支持2G,3G,4G三代网络,又加上同一代通信网络也会有不同的解决方案,所以中国国内手机通信的频率使用非常复杂。详情见表格1.

\begin{table}[!htbp]
  \centering
\caption{中国通信公司频率使用表}
\begin{tabular}{|c|c|c|c|c|}
\hline
服务提供商&通信标准&通信技术&上行频率/MHz&下行频率/MHz\\
\hline
\multirow{6}*{中国移动}&\multirow{2}*{2G}&GSM800&885-909&930-954\\
\cline{3-5}
& &GSM1800&1710-1725&1805-1820\\
\cline{2-5}
&3G&TD-SCDMA&2010-2025&2015-2025\\
\cline{2-5}
&\multirow{3}*{4G}&\multirow{3}*{LD-LTE}&1880-1890&1880-1890\\
\cline{4-5}
& & &2320-2370&2320-2370\\
\cline{4-5}
& & &2575-2635&2575-2635\\
\hline
\multirow{6}*{中国联通}&\multirow{2}*{2G}&GSM800&909-915&954-960\\
\cline{3-5}
& &GSM1800&1745-1755&1840-1850\\
\cline{2-5}
&3G&WCDMA&1940-1955&2130-2145\\
\cline{2-5}
&\multirow{3}*{4G}&\multirow{2}*{TD-LTE}&2300-2320&2300-2320\\
\cline{4-5}
&&&2555-2575&2555-2575\\
\cline{3-5}
&&FDD-LTE&1755-1765&1850-1860\\
\cline{3-5}
\hline
\multirow{5}*{中国电信}&2G&CDMA&909-915&954-960\\
\cline{2-5}
&3G&CDMA2000&1920-1935&2110-2125\\
\cline{2-5}
&\multirow{3}*{4G}&\multirow{2}*{TD-LTE}&2370-2390&2370-2390\\
\cline{4-5}
&&&2635-2655&2635-2655\\
\cline{4-5}
&&FDD-LTE&1765-1780&1860-1875\\
\hline
\end{tabular}
\end{table}

\section{电磁波的发射与传播}
\subsection{什么是地波,天波,视距传播和超视距传播?}
\subsubsection{地波}
沿陆地/海洋表面传播的电磁波,距离几百千米至几千千米,主要使用长波。
\subsubsection{天波}
天波是指利用电离层反射或折射回地球进行传播的电磁波,距离可超过1万千米,主要使用短波频段。由于电离层容易受太阳活动影响,所以天波信号不够稳定。
\subsubsection{视距传播}
视距传播使用超短波,微波作地面通信和广播,其传播距离与地面上人的视线距离相仿,一般不超过50km。
\subsubsection{超视距传播}
电磁波除了利用电磁波反射实现超视距传播,还可以利用低层大气的分层特征通过反射和折射实现超视距传输。
\section{电磁波与媒质和物体的作用}
\subsection{请解释电磁波/光波的极化}
电磁波的极化即为电场矢量末端随时间变化的特性。右手拇指指向传播方向,如果电场矢量旋转方向与四指弯曲方向一致,称为右手极化;相反则称为左手极化;若电场矢量方向不旋转则为线极化。可分为:
\begin{itemize}
  \item 线极化:电场矢量方向不变,但大小可变;
  \item 圆极化:电场矢量大小不变,但方向随时间旋转;
  \item 椭圆极化:电场矢量的大小和方向都随时间变化。
\end{itemize}

\section{电磁波作为信息的载体-信号调制}
\subsection{通信信号调制一般有哪些?}
\subsubsection{模拟信号调制}
\begin{itemize}
\item 幅度调制(AM)
\item 频率调制(FM)
\item  角度调制
\end{itemize}
\subsubsection{数字信号调制}
\begin{itemize}
\item  二进制幅度键控(BASK)
\item  二进制频移键控(BFSK)
\item  四进制频度键控(QFSK)
\item  二进制相移键控(BPSK)
\item  四进制相移键控(QPSK)
\item  正交调制(QAM)
\end{itemize}
\clearpage
\part{电磁频谱与应用综述}
电磁波看不见摸不着,但又无处不在。不同电磁波有着不同的频率,也有着不一样的波长,这导致了它们有着不同性质。而这些截然不同的性质,进而决定了它们有着完全不同的应用场景。从通信到地球尺度的探测,再到微小的钢管探伤,都能看到电磁波的身影。下文就将按照低频到高频的顺序,介绍不同波长电磁波的相关性质及其应用。
\setcounter{section}{0}
\section{极低频(Extremely low frequency)}
极低频电磁波指的是频率为3Hz到30Hz,波长为10000公里到100000公里电磁波。这种电磁波波长极长,远远大于普通地表起伏的物理尺度。因此,这种电磁波可以轻松绕过地表障碍物,实现地表远距通信。此外,极低频电磁波还可以被电离层反射,这就使得地表和电离层构成了一个巨型波导,使得该电磁波甚至有能力实现全球通信。此外,由于很少有分子的振动频率低至几十Hz,所以这种电磁波在空气和水中的衰减都很小\cite{Jursa1985Handbook},因此该电磁波还曾被用于潜艇通信\cite{Barr2000ELF},但由于其带宽实在有限,所以能传递的信息很少,现在已经不再使用。
该电磁波由于带宽太窄,所以多少应用方面的研究,但其传播距离很广的特点却引起了不少健康方面的关注。目前的研究成果表面,人长期暴露在有一定强度的ELF波内,可能会对神经系统和肝脏等组织造成一定影响。\cite{赵龙宇2012工作场所极低频电磁辐射对作业人员健康状况影响的调查分析,刘欣2013极低频电磁场暴露对从业人员肝脏功能的影响}。
\section{超低频(Super low frequency)}
超低频指的是频率介于30Hz到300Hz间的电磁波,常见电力传输所使用的即为该波段的电磁波,这也是该电磁波最常见的用途。与ELF类似的,该波段电磁波在水中传播的衰减也很微弱,因此也被用于潜艇通信,但该波段的研究主要集中于电力传输等强电方面。
\section{特低频(Ultra low frequency)}
特低频指的是300Hz到3kHz的电磁波。这个频段在地球物理学科十分有用,因为不论是地震\cite{Thomas2009On}还是等离子层扰动都会产生特低频电磁波,也就可以通过检测这些电磁波实现对地球物理状态的监控。比如,该波段电磁波在地壳中的传播速率与物质密度息息相关,只要安排一定数量的电磁波监测站,就能借助接收到的电磁波不同而推算出地壳状态。\\
该波段电磁波还被北约用于早期的山地通信\cite{Jan2001AGARD}。由于其带宽比较有限,所以现在已经不再军用,但依然被广泛采用于矿井等特殊工作场所。
\section{甚低频(Very low frequency)}
甚低频指的是3kHz到30kHz的电磁波。与ELF波类似,该波段的电磁波一样可以绕过山地,被电离层反射,实现远距信息传输。也因此更容易受太阳活动影响,远距传输时噪声很大\cite{Ghosh2002Electromagnetic}。\\
由于其具有一定带宽而又可以传播很远的距离,除了被用于最基本的声音信号通信外,也常用于导航和授时。此外,该波段也被用于潜艇通信,但通信距离不如SLF那么深,对发射机的功率也有一定的要求,但其使用频移键控调制方式时,传输速率最高能到75bit/s,已具有传递简单战术命令的命令的能力,和ELF只能传递"入/出海底"的指令相比又进了一大步。
\section{低频(Low frequency)}
低频电磁波频率为30kHz到300kHz。由于这个波段的电磁波衰减比较慢,而带宽又足够宽,能够调制具有一定带宽的语音信号,故最常用于AM无线电广播。与VLF类似,该波段也被用于授时系统和导航系统。比如我国呼号为BPL的长波授时台就可以提供微秒量级的授时服务,服务范围覆盖我国陆地全境和近海海域。\\
此外,该频段电磁波还被广泛应用于无线电身份识别领域,相关技术被称为LF RFID。这些RFID系统由两部分组成,一部分是用于标记信息的无线电标签,另外一部分是用于读取标签的读取器。这些系统的工作距离有长有短,但大部分都只局限于10cm这一量级,还由于低频信号无法高速传输数据,这也导致LF RFID无法支持高复杂度的身份识别算法\cite{Weis2011RFID}。
\section{中频(Medium frequency)}
中频电磁波位于300kHz到3MHz的范围内。与LF,VLF类似,该波段也用于授时,导航与广播。该波段的电磁波能够传递一定的距离但又不会过远,发射设备和接收设备都可以做到轻量化和低能耗,所以非常适合用于海上船只间的通信和陆地与海上船只间的通信。\\
该波段电磁波一样会被电离层反射,实现超视距通信,但通信状态极其容易受电离层状态影响。当电磁波被波动较大的D层反射时,就会带来大量噪声,超视距通信效果将大打折扣;而到了夜晚,随着太阳活动对相应区域电离层的影响减弱,该电磁波就能被较为稳定的F层反射,并实现效果非常优秀的超视距通信\cite{Ghosh2002Electromagnetic}。
\section{高频(High frequency)}
高频为3MHz到30MHz的范围内。该波段主要用于需要较大带宽和具有一定距离的通信场景,常见的有军用短波通信系统,地空通信系统,短波区域广播,岸舰通信系统,超视距雷达,全球海上遇险和安全系统。但是,这些通信系统大多使用天波的通信方式,这就使得通信质量会随着季节,太阳活动,甚至昼夜变化而发生很大改变\cite{Ghosh2002Electromagnetic}。
\section{甚高频(Very high frequency)}
甚高频指的是频率位于30MHz到300MHz的无线电波。与高频信号类似,甚高频也被广泛应用于航空通信,航海通信以及业余无线电通信。\\
由于甚高频电磁波能携带比高频电磁波多很多的信息,所以除了用于声音广播外,还被用于电视广播。但是,甚高频电磁波波长较短,所以很容易因为较大障碍物的遮挡而不能很好地传输,这就对发射天线的架设位置提出了很高的要求。而且,甚高频只有部分低频波段可以被电离层反射,所以也很难使用超视距传播技术实现远距传输,这在一定程度上限制了该无线电波的使用,但是,该电磁波在地面也有约160km的传输距离,所以作为城镇范围内的广播也已经绰绰有余了\cite{Seybold2005Introduction}。\\
由于该波段电磁波波长较短,天线尺寸已经小到可以由单人携带,基于该特点设计出了无线对讲机等个人通信设备。
\section{特高频(Ultra high frequency)}
特高频指的是300MHz到3GHz的电磁波,波长已经小于1m,所以天线尺寸能做得比较小,适合作为移动通信频段。GPS,Wi-Fi,蓝牙等使用的都是该频段电磁波。\\
但也因为无线电波波长很短,所以很容易被较大的障碍物遮挡,然而该电磁波具有穿透墙壁的能力,所以并不妨碍该电磁波作为室内通信波段。该电磁波在空气中衰减得很快,即使在没有遮挡的环境下也只能传播60km左右,若再考虑上城市环境下的的建筑物遮挡,该电磁波的传播距离会更加有限,所以适合用于组建蜂窝移动通信网络。\\
该电磁波不能被电离层反射,但可以沿着对流层传播,所以可以借助该特性实现较远距离的特高频传输。然而对流层气象状况非常不稳定,所以有效距离和传输效果波动都很大。
\section{超高频(Super high frequency)}
超高频指的是3GHz到30GHz的电磁波,波长处于厘米量级,正因如此,该波段的天线能够做得非常小,常被用于点对点通信系统,数据链和雷达系统。\\
该波段电磁波不能在电离层和地面间反复反射,无法实现超视距传输。该波段电磁波是最低频率的具有窄波束导向的电磁波,这使其具有大功率传播而不会对周围的通信系统造成严重干扰的能力。
\section{极高频(Extremely high frequency)}
极高频电磁波频率为30GHz到300GHz,波长为毫米量级,具有非常好的指向性,但由于其与大气中大部分分子的振荡频率相近,所以传播距离非常有限,不能实现大气层内远距通信。但也正由于其可以与大气分子相互作用,所以极高频电磁波被用于制造气象雷达,通过检测雷达反射波实现对大气状态的检测。\\
虽然该频段电磁波传播距离很短,但其带宽很宽,能够很容易地实现高速率信息传递,常被用于无线动态数据链架设。比如美国用于城市安全的CableFree\ MMW数据链路就具有高达1Gbit/s的超高速率。\\
由于该频段电磁波能够和水分子作用,提高水温,因此该电磁波也被用于制造非致命性微波武器,让被照射到的人感到无法忍受的痛苦却不会有生命危险,已有被称为CIWS的微波防暴设备用于远距无损伤地控制暴乱人群。
\section{太赫兹(Terahertz radiation)}
太赫兹为300GHz到3THz的电磁波,该频段电磁波位于光学和无线电的交界处,所以拥有很多特殊的性质。太赫兹的波长很小,有着不输可见光的分辨率,而且太赫兹还具有一定的穿透能力,所以在成像方面有着不错的潜力。可以在制造业中作为瑕疵探测的重要工具,又由于太赫兹波与X光不同,不会对被探测物质造成损坏,所以也适合用于生物成像,文物鉴定等特殊领域。\\
又由于该电磁波具有很高的额频率,所以即使只利用1\%的频率也可以有高达30GHz的带宽,这是其他频率电磁波无法比拟的。因此太赫兹还非常适合用于高数据容量的近距离通信,实现超小区域内的无线高速数据共享。

\end{CJK}
\bibliographystyle{plain}
\bibliography{ref.bib}

\end{document}
