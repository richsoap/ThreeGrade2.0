\documentclass{article}
% TITLE PAGE CONTENT %%%%%%%%%%%%%%%%%%%%%%%%
% Remember to fill this section out for each
% lab write-up.
%%%%%%%%%%%%%%%%%%%%%%%%%%%%%%%%%%%%%%%%%%%%%
\usepackage{CJK}
\usepackage{float}
\usepackage{subfig}
\usepackage{graphicx}
\usepackage{listings} % For source cod
\usepackage{xcolor}
\usepackage{geometry}
\geometry{left=1.7cm,right=1.7cm,top=2.0cm,bottom=2.0cm}
\usepackage{float}
\usepackage{subfig}
% END TITLE PAGE CONTENT %%%%%%%%%%%%%%%%%%%%
\newcommand{\ifc}{$I^2C$}

\begin{document}  % START THE DOCUMENT!

\begin{CJK}{UTF8}{gkai}
\title{实验八 字符型背光液晶显示屏}
\author{杨庆龙 \\1500012956}
\date{2018.5.16}
\maketitle

\section{实验目的}
\begin{enumerate}
  \item 了解字符型液晶显示模块的使用方法
  \item 掌握并行端口模拟接口时序的方法
\end{enumerate}

\section{实验原理}
\subsection{硬件接口}
  \begin{enumerate}
    \item 1:VSS接地
    \item 2:VDD接5V电源
    \item 3:V0用于对比度调节
    \item 4:RS寄存器,高电平时选择数据,低电平时选择指令
    \item 5:RW寄存器,高电平为读操作,低电平为写操作
    \item 6:使能端
    \item 7-14:双向数据线
    \item 15-16:空引脚

  \end{enumerate}

\subsection{指令}
\begin{enumerate}
  \item 0x38:表示8位地址总线,双行显示
  \item 0x08:关闭显示
  \item 0x01:表示清屏
  \item 0x06:表示右移光标,文字不平移
  \item 0x0C:打开显示,不显示 光标
  \item 0x80:设置数据储存器的位置
  \item 0x02:光标位置复位
\end{enumerate}

\end{table}


\end{CJK}

\end{document} % DONE WITH DOCUMENT!
