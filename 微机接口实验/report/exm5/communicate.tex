\documentclass{article}
% TITLE PAGE CONTENT %%%%%%%%%%%%%%%%%%%%%%%%
% Remember to fill this section out for each
% lab write-up.
%%%%%%%%%%%%%%%%%%%%%%%%%%%%%%%%%%%%%%%%%%%%%
\usepackage{CJK}
\usepackage{listings} % For source code
% END TITLE PAGE CONTENT %%%%%%%%%%%%%%%%%%%%


\begin{document}  % START THE DOCUMENT!

\begin{CJK}{UTF8}{gkai}
\title{实验五 单片机串行通信}
\author{杨庆龙 \\1500012956}
\date{2018.4.18}
\maketitle
\section{实验目的}
\begin{itemize}
  \item 了解串行通信的基本知识
  \item 掌握用单片机串行口实现串行通信的方法
\end{itemize}
\section{实验原理}
\subsection{串行通信的异步和同步传送方式}
CPU与外设的基本通信方式可分为并行通信和串行通信两类。并行是指要传输的的按照二进制位同时传输,串行则是逐位传输的方式。\\
串行传输所用的传输线远少于并行传输线,也是实际常用的传输线。\\
单片机使用异步传输的通信方式,其特点为
\begin{itemize}
  \item 数据是离散发送的
  \item 通信双方时钟频率相同
  \item 通信双方按照异步通信协议传输字符
\end{itemize}
字符帧格式为,一个起始位,N位数据位,一个校验位,一个停止位\\
异步通信使用波特率=单帧位数*每秒的帧数表示数据传输的速率。并使用高于传输速率的时钟进行采样提高采样的准确性。
\subsection{MCS-51的串行通信接口}
MCS51内部有一个可编程的全双工串行通信口,可作为通用异步接收和发送器,也可作为同步移位寄存器使用。内部的串行通信口,有两个独立的接收发送缓冲器SBUF,对外也有两条独立的收发信号线RxD和TxD。可以同时发送,接收数据,实现全双工传送。与串行通信有关的寄存器有多个,用SCON控制和监视串行口的工作状态。
\begin{table}
  \centering
  \caption{串行控制寄存器SCON}
  \begin{tabular}{|c|c|c|c|c|c|c|c|}
    \hline
    D7&D6&D5&D4&D3&D2&D1&D0\\
    \hline
    SM0&SM1&SM2&REN&TB8&RB8&TI&RI\\
    \hline
  \end{tabular}
\end{table}
\begin{itemize}
  \item SM0,SM1:00_同步移位寄存器,01_8位UART,10_9位UART,11_9位UART可变波特率
  \item REN:允许接收控制位,由软件置位或清除
  \item TB8:模式2,33中的第九位
  \item RB8:该位是模式2和3中已接收的第九位
  \item TI:发送中断标识
  \item RI:接收中断标识
\end{itemize}
\section{实验内容}
\subsection{观察UART通信波形}
设定串行口工作方式1,用1200bps循环发送一个字节55H或8AH,用示波器观察TxD的电平和信号结构,给出1200bps波特率条件下的TH1计算值,码元宽度的计算值和测量值。
\subsection{串口收发实验}
编写一个程序是单片机通过键盘输入数据,再把数据送到PC上,PC又将数据发送给单片机,单片机将数据现实到数码管上。
\subsection{串口作为STDIO}
初始化了UART0后,单片机会将该串口作为STDIO,用户可以直接用printf和scanf从串口中读写数据。编写程序通过串口输入数据,进行四则运算后输出到串口。



\end{CJK}

\end{document} % DONE WITH DOCUMENT!
