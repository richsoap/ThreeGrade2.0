\documentclass{article}
% TITLE PAGE CONTENT %%%%%%%%%%%%%%%%%%%%%%%%
% Remember to fill this section out for each
% lab write-up.
%%%%%%%%%%%%%%%%%%%%%%%%%%%%%%%%%%%%%%%%%%%%%
\usepackage{CJK}
\usepackage{float}
\usepackage{subfig}
\usepackage{graphicx}
\usepackage{listings} % For source cod
\usepackage{xcolor}
\usepackage{geometry}
\geometry{left=1.7cm,right=1.7cm,top=2.0cm,bottom=2.0cm}
\usepackage{float}
\usepackage{subfig}
% END TITLE PAGE CONTENT %%%%%%%%%%%%%%%%%%%%
\newcommand{\ifc}{$I^2C$}

\begin{document}  % START THE DOCUMENT!

\begin{CJK}{UTF8}{gkai}
\title{实验七 \ifc总线}
\author{杨庆龙 \\1500012956}
\date{2018.5.9}
\maketitle

\section{实验目的}
\begin{enumerate}
  \item 了解\ifc总线的基本原理
  \item 掌握使用C8051F020作为主设备进行\ifc通信的方法
  \item 了解时钟芯片DS1307的使用方法
\end{enumerate}

\section{实验原理}
\subsection{\ifc总线的基本原理}
\subsubsection{物理结构}
\ifc的总线标准有两条信号,其中SDA为双向数据信号,SCL为双向时钟信号。
数据信号只在时钟信号为低时变化,在时钟上升沿采样。控制信号在时钟信号的为高时变化,下降沿表示起始位,上升沿表示停止位。
\subsubsection{通信模式}
\ifc数据传输由主设备产生一个起始位开始,如何传递7个地址位,指定操作设备。接下来一位是读写位,0表示写入,1表示读入。接下来由从设备驱动,如果从设备地址与主设备地址相同,从设备将把SDA信号拉低,表示确认传输。
\subsection{控制寄存器}
\begin{table}
  \caption{SMB控制寄存器}
  \label{pro1_2}
  \begin{tabular}{|c|c|c|c|c|c|c|c|}
    \hline
    D7&D6&D5&D4&D3&D2&D1&D0 \\
    \hline
    busy&ensmb&sta&sto&si&aa&fte&toe\\
    \hline
  \end{tabular}
\end{table}
\begin{itemize}
  \item BUSY:11表示设备忙
  \item ENSMB:1表示设备激活
  \item STA:起始位控制,设置为1时,会发送起始位。
  \item STO:阶数为控制,设置为1时,会发送停止位。
  \item SI:中断标识
  \item AA:应答标识
  \item FTE:定时总线释放
  \item TOE:更短时间的定时总线释放
\end{itemize}

SMB0DAT寄存器用于存放即将发送的\ifc数据,软件只有在SI位有效的情况下才能访问这个寄存器。SMB0ADR寄存器用于存放设备的地址
\section{实验内容}
\subsection{显示时钟}
SMB的SDA和SCL与SPI的部分管脚共用,因此需要检测跳线是否接到了偏下的两个接线柱。

\section{源码}

\begin{lstlisting}[language=C,numbers=left,numberstyle=\tiny,%frame=shadowbox,
   rulesepcolor=\color{red!20!green!20!blue!20},
   keywordstyle=\color{blue!70!black},
   commentstyle=\color{blue!90!},
   basicstyle=\ttfamily]
   #include <C8051F020.h>
   #include <stdio.h>
   #include "../../includes/time.h"
   #include "../../includes/i2c.h"
   #include "../../includes/communicate.h"

   void smbus_int(void) interrupt 7 {
     bit FAIL = 0;
     static unsigned char i;
     switch(SMB0STA) {
       case SMB_START:
       case SMB_RP_START:
       SMB0DAT = smb_buf[0];
       STA = 0;
       i = 0;
       break;
       case SMB_MTADDACK:
       SMB0DAT = smb_buf[1];
       break;
       case SMB_MTDBACK:
       if(i < smb_len) {
         SMB0DAT = smb_buf[i+2];
         i ++;
       }
       else {
         result = S_OVER;
         STO = 1;
       }
       break;
       case SMB_MRADDACK:
       if(smb_len == 1)
       AA = 0;
       else
       AA = 1;
       break;
       case SMB_MRDBACK:
       if(i < smb_len) {
         smb_buf[i+1] = SMB0DAT;
         i ++;
         AA = 1;
       }
       if(i >= smb_len)
       AA = 0;
       break;
       case SMB_MRDBNACK:
       smb_buf[i+1] = SMB0DAT;
       STO = 1;
       AA = 1;
       result = R_OVER;
       break;
       case SMB_MTADDNACK:
       case SMB_MTDBNACK:
       case SMB_MTARBLOST:
       case SMB_MRADDNACK:
       FAIL = 1;
       break;
       default:
       FAIL = 1;
     }
     if(FAIL) {
       SMB0CN &= -0x40;
       SMB0CN |= 0x40;
       STA = 0;
       STO = 0;
       AA = 0;
       result = SMB_FAIL;
       FAIL = 0;
     }
     SI = 0;
   }

   void main() {
     int i;
     int counter;
     int flag;
     unsigned char value;
     unsigned char c;
     WDTCN = 0xDE;
     WDTCN = 0xAD;
     sysclk_init();
     i2c_port_init();
     uart0_init();
     TI0 = 1;
     i2c_init();
     SI = 0;
     EIE1 |= 0x02;
     EA = 1;
       printf("start!\r\n");
     smb_transmit(0,0);
     smb_receive(1);
     if (smb_buf[1] & 0x80) {
       unsigned char b = smb_buf[1];
       smb_buf[2] = b & 0x7F;
       smb_transmit(0,1);
     }
   	 i = 0;
   	 counter = 0;
     while(1) {
     	c = getchar();
   		switch(c) {
   			case 'y':
   			scanf("%02bx",smb_buf + 2);
   			    smb_transmit(6,1);
   			break;
   			case 'M':
   			scanf("%02bx",smb_buf + 2);
   			    smb_transmit(5,1);
   			break;
   			case 'd':
   			scanf("%02bx",smb_buf + 2);
   			    smb_transmit(4,1);
   			break;
   			case 'h':
   			scanf("%02bx",smb_buf + 2);
   			    smb_transmit(2,1);
   			break;
   			case 'm':
   			scanf("%02bx",smb_buf + 2);
   			    smb_transmit(1,1);
   			break;
   			case 's':
   			scanf("%02bx",smb_buf + 2);
   			    smb_transmit(0,1);
   			break;
   			case 'p': flag = 1;
   		}
   		if(flag)
   		break;
     }
     while(1) {
       smb_transmit(0,0);
       smb_receive(8);
       delay(100);
   	printf("%02bx-",smb_buf[7]);
       printf("%02bx-",smb_buf[6]);
       printf("%02bx ",smb_buf[5]);
       //printf("%d%d:",smb_buf[5] >> 12,(smb_buf[5] >> 8) & 0xf);
       printf("%02bx:",smb_buf[3]);
       printf("%02bx:",smb_buf[2]);
       printf("%02bx",smb_buf[1] & 0x7f);
   	printf("\r\n");
     }
   }

\end{lstlisting}

\end{CJK}

\end{document} % DONE WITH DOCUMENT!
