\documentclass{article}
% TITLE PAGE CONTENT %%%%%%%%%%%%%%%%%%%%%%%%
% Remember to fill this section out for each
% lab write-up.
%%%%%%%%%%%%%%%%%%%%%%%%%%%%%%%%%%%%%%%%%%%%%
\usepackage{CJK}
\usepackage{listings} % For source code
\usepackage{float}
\usepackage{subfig}
% END TITLE PAGE CONTENT %%%%%%%%%%%%%%%%%%%%
\newcommand{\ifc}{$I^2C$}

\begin{document}  % START THE DOCUMENT!

\begin{CJK}{UTF8}{gkai}
\title{实验七 \ifc总线}
\author{杨庆龙 \\1500012956}
\date{2018.5.9}
\maketitle

\section{实验目的}
\begin{enumerate}
  \item 了解\ifc总线的基本原理
  \item 掌握使用C8051F020作为主设备进行\ifc通信的方法
  \item 了解时钟芯片DS1307的使用方法
\end{enumerate}

\section{实验原理}
\subsection{\ifc总线的基本原理}
\subsubsection{物理结构}
\ifc的总线标准有两条信号,其中SDA为双向数据信号,SCL为双向时钟信号。
数据信号只在时钟信号为低时变化,在时钟上升沿采样。控制信号在时钟信号的为高时变化,下降沿表示起始位,上升沿表示停止位。
\subsubsection{通信模式}
\ifc数据传输由主设备产生一个起始位开始,如何传递7个地址位,指定操作设备。接下来一位是读写位,0表示写入,1表示读入。接下来由从设备驱动,如果从设备地址与主设备地址相同,从设备将把SDA信号拉低,表示确认传输。
\subsection{控制寄存器}
\begin{table}
  \caption{SMB控制寄存器}
  \label{pro1_2}
  \begin{tabular}{|c|c|c|c|c|c|c|c|}
    \hline
    D7&D6&D5&D4&D3&D2&D1&D0 \\
    \hline
    busy&ensmb&sta&sto&si&aa&fte&toe\\
    \hline
  \end{tabular}
\end{table}
\begin{itemize}
  \item BUSY:11表示设备忙
  \item ENSMB:1表示设备激活
  \item STA:起始位控制,设置为1时,会发送起始位。
  \item STO:阶数为控制,设置为1时,会发送停止位。
  \item SI:中断标识
  \item AA:应答标识
  \item FTE:定时总线释放
  \item TOE:更短时间的定时总线释放
\end{itemize}

SMB0DAT寄存器用于存放即将发送的\ifc数据,软件只有在SI位有效的情况下才能访问这个寄存器。SMB0ADR寄存器用于存放设备的地址
\section{实验内容}
\subsection{显示时钟}
SMB的SDA和SCL与SPI的部分管脚共用,因此需要检测跳线是否接到了偏下的两个接线柱。





\end{CJK}

\end{document} % DONE WITH DOCUMENT!


\\
学校里,提供二维码,为需要物品打标签。
钥匙打码
可选打码
