\documentclass{article}
%%%%%%%%%%%%%%%%%%%%%%%%%%%%%%%%%%%%%%%%%%%%%%%
% An example of a lab report write-up.
%%%%%%%%%%%%%%%%%%%%%%%%%%%%%%%%%%%%%%%%%%%%%%
% This is a combination of several labs that I have done in the past for
% Computer Engineering, so it is not to be taken literally, but instead used as
% a great starting template for your own lab write up.  When creating this
% template, I tried to keep in mind all of the functions and functionality of
% LaTeX that I spent a lot of time researching and using in my lab reports and
% include them here so that it is fairly easy for students first learning LaTeX
% to jump on in and get immediate results.  However, I do assume that the
% person using this guide has already created at least a "Hello World" PDF
% document using LaTeX (which means it's installed and ready to go).
%
% My preference for developing in LaTeX is to use the LaTeX Plugin for gedit in
% Linux.  There are others for Mac and Windows as well (particularly MikTeX).
% Another excellent plugin is the Calc2LaTeX plugin for the OpenOffice suite.
% It makes it very easy to create a large table very quickly.
%
% Professors have different tastes for how they want the lab write-ups done, so
% check with the section layout for your class and create a template file for
% each class (my recommendation).
%
% Also, there is a list of common commands at the bottom of this document.  Use
% these as a quick reference.  If you'd like more, you can view the "LaTeX Cheat
% Sheet.pdf" included with this template material.
%
% (c) 2009 Derek R. Hildreth <derek@derekhildreth.com> http://www.derekhildreth.com
% This work is licensed under the Creative Commons Attribution-NonCommercial-ShareAlike License. To view a copy of this license, visit http://creativecommons.org/licenses/by-nc-sa/1.0/ or send a letter to Creative Commons, 559 Nathan Abbott Way, Stanford, California 94305, USA.
%%%%%%%%%%%%%%%%%%%%%%%%%%%%%%%%%%%%%%%%%%%%%%
\documentclass[aps,letterpaper,10pt]{article}
\input kvmacros % For Karnaugh Maps (K-Maps)

\usepackage{graphicx} % For images
\usepackage{float}    % For tables and other floats
\usepackage{verbatim} % For comments and other
\usepackage{amsmath}  % For math
\usepackage{amssymb}  % For more math
\usepackage{fullpage} % Set margins and place page numbers at bottom center
\usepackage{listings} % For source code
\usepackage{subfig}   % For subfigures
\usepackage[usenames,dvipsnames]{color} % For colors and names
\usepackage[pdftex]{hyperref}           % For hyperlinks and indexing the PDF
\hypersetup{ % play with the different link colors here
    colorlinks,
    citecolor=blue,
    filecolor=blue,
    linkcolor=blue,
    urlcolor=blue % set to black to prevent printing blue links
}

\definecolor{mygrey}{gray}{.96} % Light Grey
\lstset{
	language=[ISO]C++,              % choose the language of the code ("language=Verilog" is popular as well)
   tabsize=3,							  % sets the size of the tabs in spaces (1 Tab is replaced with 3 spaces)
	basicstyle=\tiny,               % the size of the fonts that are used for the code
	numbers=left,                   % where to put the line-numbers
	numberstyle=\tiny,              % the size of the fonts that are used for the line-numbers
	stepnumber=2,                   % the step between two line-numbers. If it's 1 each line will be numbered
	numbersep=5pt,                  % how far the line-numbers are from the code
	backgroundcolor=\color{mygrey}, % choose the background color. You must add \usepackage{color}
	%showspaces=false,              % show spaces adding particular underscores
	%showstringspaces=false,        % underline spaces within strings
	%showtabs=false,                % show tabs within strings adding particular underscores
	frame=single,	                 % adds a frame around the code
	tabsize=3,	                    % sets default tabsize to 2 spaces
	captionpos=b,                   % sets the caption-position to bottom
	breaklines=true,                % sets automatic line breaking
	breakatwhitespace=false,        % sets if automatic breaks should only happen at whitespace
	%escapeinside={\%*}{*)},        % if you want to add a comment within your code
	commentstyle=\color{BrickRed}   % sets the comment style
}

% Make units a little nicer looking and faster to type
\newcommand{\Hz}{\textsl{Hz}}
\newcommand{\KHz}{\textsl{KHz}}
\newcommand{\MHz}{\textsl{MHz}}
\newcommand{\GHz}{\textsl{GHz}}
\newcommand{\ns}{\textsl{ns}}
\newcommand{\ms}{\textsl{ms}}
\newcommand{\s}{\textsl{s}}

% TITLE PAGE CONTENT %%%%%%%%%%%%%%%%%%%%%%%%
% Remember to fill this section out for each
% lab write-up.
%%%%%%%%%%%%%%%%%%%%%%%%%%%%%%%%%%%%%%%%%%%%%
\newcommand{\labno}{04}
\newcommand{\labtitle}{模数与数模转换}
\newcommand{\authorname}{杨庆龙}
\newcommand{\idnumber}{1500012956}
\newcommand{\labdate}{2018.4.11}
\usepackage{CJK}
\usepackage{listings}
% END TITLE PAGE CONTENT %%%%%%%%%%%%%%%%%%%%


\begin{document}  % START THE DOCUMENT!

\begin{CJK}{UTF8}{gkai}

% TITLE PAGE %%%%%%%%%%%%%%%%%%%%%%%%%%%%%%%%%%%%%%
% If you'd like to change the content of this,
% do it in the "TITLE PAGE CONTENT" directly above
% this message
%%%%%%%%%%%%%%%%%%%%%%%%%%%%%%%%%%%%%%%%%%%%%%%%%%%

\title{实验四 模数与数模转换}
\author{杨庆龙}
\date{2018.4.11}
\maketitle

%%%%%%%%%%%%%%%%%%%%%%%%%%%%%%
%%%%%%%%%%%%%%%%%%%%%%%%%%%%%%
\section{实验目的}
%No Text Here
%%%%%%%%%%%%%%%%%%%%%%%%%%%%%%%
\begin{itemize}
	\item 了解模数和数模转换电路的原理和使用方法
	\item 掌握MCS-51系列单片机中定时器和计数器的使用方法
	\item 掌握使用示波器,信号源对单片机系统进行调试的方法
\end{itemize}
\vspace{3mm}

%%%%%%%%%%%%%%%%%%%%%%%%%%%%%%
%%%%%%%%%%%%%%%%%%%%%%%%%%%%%%
\section{实验原理}
\subsection{数模转换器}
\subsubsection{控制}
单片机系统中,有两个12位DAC和两个比较器,通过特殊功能寄存器DAC0CN实现控制,详见\ref{table_DAC0CN}
\begin{table}[!htbp]
  \centering
  \caption{DAC0CN寄存器结构}
  \label{table_DAC0CN}
\begin{tabular}{|c|c|c|c|c|c|c|c|}
  \hline
  D7&D6&D5&D4&D3&D2&D1&D0\\
  DAC0EN&&&DAC0MD1&DAC0MD0&DAC0DF2&DAC0DF1&DAC0DF0\\
	\hline
\end{tabular}
\end{table}
\begin{itemize}
  \item DAC0EN:使能设置,0禁止,1允许
  \item DAC0MD1-0:工作模式,00写入触发,01,10,11分别对应计数器3,4,2
  \item DAC0DF2-0:数据格式,000:右对齐,001右对齐左移一位..1xx左对齐
\end{itemize}
\subsubsection{参考电压}
单片机系统中,使用REF0CN寄存器进行参考电压控制,详见\ref{table_REF0CN}
\begin{table}[!htbp]
  \caption{REF0CN寄存器结构}
  \label{table_REF0CN}
  \begin{tabular}{|c|c|c|c|c|c|c|c|}
    \hline
    D7&D6&D5&D4&D3&D2&D1&D0\\
    &&&AD0VRS&AD1VRS&TEMPE&BIASE&REFBE\\
    \hline
  \end{tabular}
\end{table}
\begin{itemize}
  \item AD0VRS:ADC0的参考电压,0为VREF0,1为DAC0输出
  \item AD1VRS:ADC1的参考电压,0为VREF1,1为AV+
  \item TEMPE:内部温度传感器,0禁止,1允许
  \item BIASE:偏置电压允许位,必须设为1
  \item REFBE:内部参考电压允许位,0禁止,1允许
\end{itemize}
%%%%%%%%%%%%%%%%%%%%%%%%%%%%%%%%%%
\subsection{模数转换器}
单片机内部有一个片内ADC0,一个9通道输入多路选择开关和可编程增益放大器,提供100ksps下的真12位进度。
\subsubsection{多路选择器}
使用AMX0SL寄存器选择输入情况,0-7对应AIN0-AIN7,其余为温度传感器
\subsubsection{时钟设置}
ADC0使用ADC0CF寄存器设置SAR的时钟,详见\ref{table_ADC0CF}
\begin{table}[!htbp]
  \caption{ADC0CF寄存器结构}
  \label{table_ADC0CF}
  \begin{tabular}{|c|c|c|c|c|c|c|c|}
    \hline
    D7&D6&D5&D4&D3&D2&D1&D0\\
    ADC0SC4&ADC0SC3&ADC0SC2&ADC0SC1&ADC0SC0&AMP0GN2&AMP0GN1&AMP0GN0\\
    \hline
  \end{tabular}
\end{table}
\begin{itemize}
  \item ADC0SC:设置时钟频率为 $\frac{SYSCLK}{CLK_{SAR0}}-1$
  \item AMP0GN:设置内部电压增益,000,001,010,011为1,2,4,8,10x为16,11x为0.5
\end{itemize}
%%%%%%%%%%%%%%%%%%%%%%%%%%%%%%
\subsection{定时器2}
定时器2共有三种工作方式:带捕获的16位定时器/计数器模式,带自装载的16位定时器/计数器,串口0波特率发生器
\begin{table}[!htbp]
  \centering
  \caption{T2CON寄存器结构}
  \begin{tabular}{|c|c|c|c|c|c|c|c|}
    \hline
    D7&D6&D5&D4&D3&D2&D1&D0\\
    TF2&EXF2&RCLK0&TCLK0&EXEN2&TR2&C/T2&CP/RL2\\
    \hline
  \end{tabular}

\end{table}
%%%%%%%%%%%%%%%%%%%%%%%%%%%%%%

\subsection{定时器3}
定时器3仅可工作在自装载模式下
\begin{table}[!htbp]
  \centering
  \caption{TMR3CN寄存器结构}
  \begin{tabular}{|c|c|c|c|c|c|c|c|}
    \hline
    D7&D6&D5&D4&D3&D2&D1&D0\\
    TF3&&&&&TR3&T3M&T3XCLK\\
    \hline
  \end{tabular}

\end{table}
%%%%%%%%%%%%%%%%%%%%%%%%%%%%%%

\subsection{定时器4}
定时器4和定时器2相同
\begin{table}[!htbp]
  \centering
  \caption{TMR3CN寄存器结构}
  \begin{tabular}{|c|c|c|c|c|c|c|c|}
    \hline
    D7&D6&D5&D4&D3&D2&D1&D0\\
    TF3&&&&&TR3&T3M&T3XCLK\\
    \hline
  \end{tabular}

\end{table}
%%%%%%%%%%%%%%%%%%%%%%%%%%%%%%
\section{实验内容}
\subsection{数模转换}
% 使用DAC输出三角波,用示波器验证输入信号的频率与幅度,探头接TP_DAC,地线接TP_AGND。

\subsection{模数转换}
使用ADC进行模数转换,并将结果输出到数码管显示。

\subsection{模数数模联调}
\subsubsection{转换输出}
模数转换从信号源输入,将转换结果再通过数模转换输出,输出结果用示波器查看。
\subsubsection{音频回放}
音谱输入已经接到AIN1上(使用AMX0SL选择),音谱输出用DAC1驱动(低位0xD5,高位0xD6)。

\end{CJK}

\end{document} % DONE WITH DOCUMENT!
