\documentclass{article}
% TITLE PAGE CONTENT %%%%%%%%%%%%%%%%%%%%%%%%
% Remember to fill this section out for each
% lab write-up.
%%%%%%%%%%%%%%%%%%%%%%%%%%%%%%%%%%%%%%%%%%%%%
\usepackage{CJK}
\usepackage{float}
\usepackage{subfig}
\usepackage{graphicx}
\usepackage{listings} % For source cod
\usepackage{xcolor}
\usepackage{geometry}
\geometry{left=1.7cm,right=1.7cm,top=2.0cm,bottom=2.0cm}
\usepackage{float}
\usepackage{subfig}
% END TITLE PAGE CONTENT %%%%%%%%%%%%%%%%%%%%


\begin{document}  % START THE DOCUMENT!

\begin{CJK}{UTF8}{gkai}
\title{实验六 SPI总线}
\author{杨庆龙 \\1500012956}
\date{2018.4.25}
\maketitle
\section{实验目的}
\begin{itemize}
  \item 了解SPI总线的基本时许
  \item 了解串行FLASH芯片的基本原理
  \item 掌握串行FLASH芯片的基本用法
\end{itemize}
\section{实验原理}
\subsection{SPI总线简介}
\subsubsection{传输线}
\begin{itemize}
  \item MOSI:主设备到从设备的数据线
  \item MISO:从设备到主设备的数据线
  \item SPCK:主设备驱动的时钟信号
  \item NSS:从设备选择线
\subsubsection{优点}
\begin{itemize}
  \item 全双工
  \item 协议灵活
  \item 接口简单
  \item 信号单向传输
\end{itemize}
\subsubsection{缺点}
\begin{itemize}
  \item 管脚较多
  \item 没有流控制信号,没有应答机制
  \item 只有一个主设备
  \item 数据传输距离比较近
\end{itemize}
配置寄存器如表\ref{table1}
\begin{table}
  \caption{SPI配置寄存器 SPI0CFG}
  \label{table1}
  \begin{tabular}{|c|c|c|c|c|c|c|c|}
    \hline
    D7&D6&D5&D4&D3&D2&D1&D0\\
    \hline
    CKPHA&CKOPL&BC2&BC1&BC0&SPIFRS2&SPIFRS1&SPIFRS0\\
    \hline
  \end{tabular}
\end{table}

\subsection{SPI接口}
C8051F020的SPI控制器可工作在主模式或从模式下,相关控制使用XBR设置,各寄存器功能如下
\begin{itemize}
  \item CKPHA:SPI时钟相位
  \item CKPOL:SPI时钟极性
  \item BC2-0:获得当前帧已发送的比特数
  \item SPIFR2-0:用来设置帧大小
  \item SPIF:中断标识,软清除
  \item WCOL:写入碰撞位,软清除
  \item MODF:主模式碰撞位,软清除
  \item RXOVRN:接收溢出,软清除
  \item TXBSY:发送忙标识,自动清除
  \item SLVSEL:选中标识,NSS为低时置1
  \item MSTEN:主模式允许位
  \item SPIEN:SPI允许位
\end{itemize}
使用SPI0CKR设置SPI时钟频率
$$f=\frac{SYSCLK}{2\times\left\(SPI0CKR+1\right\)}$$
\subsection{SPI Flash的使用}
\section{思考题}
\begin{enumerate}
  \item 还是保持03FH,但如果写入其他值,就可能会出问题。因为flash只能写入0,不能写入bit1,所以当我们要在同一个位置写两次时,就可能会出问题。
  \item 可能会得到错误的结果,如果实际数据长度比所需数据长,那只能得到部分bit,如果更短,则会用后续数据补上去。都会影响后续数据的读入。
\end{enumerate}
\section{源码}
\begin{lstlisting}[language=C,numbers=left,numberstyle=\tiny,%frame=shadowbox,
   rulesepcolor=\color{red!20!green!20!blue!20},
   keywordstyle=\color{blue!70!black},
   commentstyle=\color{blue!90!},
   basicstyle=\ttfamily]
   #include <C8051F020.h>
   #include <stdio.h>
   #include "time.h"
   #include "SPI.h"
   #include "communicate.h"

   void main() {
     unsigned char v;
     unsigned char c;
     unsigned int addr;
     unsigned int v1;
     unsigned int v2;
     WDTCN = 0xDE;
     WDTCN = 0xAD;
     sysclk_init();
     P6 = 0x80;
     uart0_init();
     spi_init();
     while(1){
       do {
         c = getchar();
       }while((c == '') || (c == '\r') || (c == '\n'));
       scanf("%lx", &addr);
       switch(c) {
         case 'd':
         ///////////////////////
         break;
         case 'w':
         scanf("%bx", &v1);
         flash_write(addr,v1);
         printf("\r\n W %lx %bx OK\r\n",addr,v1);
         break;
         case 'c':
         flash_erease(addr);
         printf("\r\n C %lx OK\r\n",addr);
         break;
       }
     }
   }

\end{lstlisting}
\end{CJK}

\end{document} % DONE WITH DOCUMENT!


\\
学校里,提供二维码,为需要物品打标签。
钥匙打码
可选打码
