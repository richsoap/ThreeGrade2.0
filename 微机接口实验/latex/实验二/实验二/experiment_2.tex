%!TEX program = xelatex
%!TEX program = xelatex
\documentclass[12pt,a4paper]{article}
\usepackage{xeCJK}
\usepackage[utf8]{inputenc}
\usepackage[T1]{fontenc}

\usepackage{amsmath,amssymb,array}
\usepackage{bm,booktabs}
\usepackage{capt-of,caption,color}
\usepackage{diagbox}
\usepackage{enumerate,enumitem}
\usepackage{fancyhdr,fixltx2e,float,fontspec}
\usepackage{geometry,graphicx,grffile}
\geometry{left=2.2cm,right=2.2cm,top=3.5cm,bottom=3cm}
\usepackage{hyperref}
\hypersetup{hidelinks}
\usepackage{indentfirst}
\usepackage{listings,longtable}
\newcommand{\tabincell}[2]{\begin{tabular}{@{}#1@{}}#2\end{tabular}}
\usepackage{makecell,multirow}
\usepackage{pifont}
\usepackage{rotating}
\usepackage{setspace,subfigure}
\usepackage{textcomp,tikz,titlesec}
\usepackage{wrapfig}
\usepackage{xltxtra,xunicode}

\usepackage{tikz,mathpazo}
\usetikzlibrary{shapes.geometric, arrows}
\tikzstyle{startstop} = [rectangle, rounded corners, minimum width = 2cm, minimum height=1cm,text centered, draw = black, fill = red!40]
\tikzstyle{io} = [trapezium, trapezium left angle=70, trapezium right angle=110, minimum width=2cm, minimum height=1cm, text centered, draw=black, fill = blue!40]
\tikzstyle{process} = [rectangle, minimum width=3cm, minimum height=1cm, text centered, draw=black, fill = yellow!50]
\tikzstyle{decision} = [diamond, aspect = 3, text centered, draw=black, fill = green!30]
% 箭头形式
\tikzstyle{arrow} = [->,>=stealth]

\lstset{%  
%alsolanguage=Java,  
language={Verilog},       %language为,还有{[Visual]C++}  
%alsolanguage=[ANSI]C,      %可以添加很多个alsolanguage,如alsolanguage=matlab,alsolanguage=VHDL等  
%alsolanguage= tcl,  
%alsolanguage= XML,  
tabsize=4, %  
  frame=shadowbox, %把代码用带有阴影的框圈起来  
  commentstyle=\color{red!50!green!50!blue!50},%浅灰色的注释  
  rulesepcolor=\color{red!20!green!20!blue!20},%代码块边框为淡青色  
  keywordstyle=\color{blue!90}\bfseries, %代码关键字的颜色为蓝色,粗体  
  showstringspaces=false,%不显示代码字符串中间的空格标记  
  stringstyle=\ttfamily, % 代码字符串的特殊格式  
  keepspaces=true, %  
  breakindent=22pt, %  
  numbers=left,%左侧显示行号 往左靠,还可以为right,或none,即不加行号  
  stepnumber=1,%若设置为2,则显示行号为1,3,5,即stepnumber为公差,默认stepnumber=1  
  %numberstyle=\tiny, %行号字体用小号  
  numberstyle={\color[RGB]{0,192,192}\tiny} ,%设置行号的大小,大小有tiny,scriptsize,footnotesize,small,normalsize,large等  
  numbersep=8pt,  %设置行号与代码的距离,默认是5pt  
  basicstyle=\footnotesize, % 这句设置代码的大小  
  showspaces=false, %  
  flexiblecolumns=true, %  
  breaklines=true, %对过长的代码自动换行  
  breakautoindent=true,%  
  breakindent=4em, %  
%  escapebegin=
%\begin{CJK*}{GBK}{hei},escapeend=\end{CJK*}
%,  
  aboveskip=1em, %代码块边框  
  tabsize=2,  
  showstringspaces=false, %不显示字符串中的空格  
  backgroundcolor=\color[RGB]{245,245,244},   %代码背景色  
  %backgroundcolor=\color[rgb]{0.91,0.91,0.91}    %添加背景色  
  escapeinside='',  %在``里显示中文  
  %% added by http://bbs.ctex.org/viewthread.php?tid=53451  
  fontadjust,  
  captionpos=t,  
  framextopmargin=2pt,framexbottommargin=2pt,abovecaptionskip=-3pt,belowcaptionskip=3pt,  
  xleftmargin=2em,xrightmargin=2em, % 设定listing左右的空白  
  texcl=true,  
  % 设定中文冲突,断行,列模式,数学环境输入,listing数字的样式  
  extendedchars=false,columns=flexible,mathescape=true  
  % numbersep=-1em  
}  

\usepackage{tabularx}
\lstset{%
language={C},       %language为Verilog,还有{[Visual]C++}
%alsolanguage=[ANSI]C,      %可以添加很多个alsolanguage,如alsolanguage=matlab,alsolanguage=VHDL等
tabsize=4, %
  showstringspaces=false,%不显示代码字符串中间的空格标记
  stringstyle=\ttfamily, % 代码字符串的特殊格式
  numbers=left,%左侧显示行号 往左靠,还可以为right,或none,即不加行号
  numberstyle={\color[RGB]{0,192,192}\tiny} ,%设置行号的大小,大小有tiny,scriptsize,footnotesize,small,normalsize,large等
  numbersep=8pt,  %设置行号与代码的距离,默认是5pt
  basicstyle=\footnotesize, % 这句设置代码的大小
  showspaces=false, %
  flexiblecolumns=true, %
  breaklines=true, %对过长的代码自动换行
  showstringspaces=false, %不显示字符串中的空格
  escapeinside='',  %在``里显示中文
  framextopmargin=2pt,framexbottommargin=2pt,abovecaptionskip=-3pt,belowcaptionskip=3pt,
  xleftmargin=2em,xrightmargin=2em, % 设定listing左右的空白
  texcl=true,
  % 设定中文冲突,断行,列模式,数学环境输入,listing数字的样式
  extendedchars=false,columns=flexible,mathescape=false
}

%%%%%%%%%%%%%%%%%%%%%%%%%%%%%%%%%%%%%%%%%%%%%%%%%%%%%%%%%%%%%%%%%%%%%%%%%%%%%%%%%%%%
%页眉页脚
\pagestyle{fancy}
\lhead{\footnotesize 实验四:模数和数模转换}
\chead{}
\rhead{\footnotesize \leftmark}
\lfoot{}
\cfoot{}
\rfoot{\thepage}

%%%%%%%%%%%%%%%%%%%%%%%%%%%%%%%%%%%%%%%%%%%%%%%%%%%%%%%%%%%%%%%%%%%%%%%%%%%%%%%%%%%%
%标题页
\title{\bf\LARGE 实验四 \ \ 模数和数模转换}
\author{杨庆龙 \ \ \ \ 1500012956}
\date{2018年4月11日}
\setcounter{page}{0}
\renewcommand \contentsname{\Large 目录}
\renewcommand \today{\number \year 年 \number \month 月 \number \day 日}
%\titleformat{\title}{\song}{\thetitle}{1em}{}

\titleformat{\section}{\centering\Large\bfseries}{\S\,\thesection}{1em}{}
\linespread{1.3} %行高
%%%%%%%%%%%%%%%%%%%%%%%%%%%%%%%%%%%%%%%%%%%%%%%%%%%%%%%%%%%%%%%%%%%%%%%%%%%%%%%%%%%%
%正文
\begin{document}
\begin{spacing}{1.3}  %段间距,公式行间距
\newgeometry{top=5cm}
\maketitle
\section{实验目的}
%No Text Here
%%%%%%%%%%%%%%%%%%%%%%%%%%%%%%%
\begin{itemize}
	\item 了解模数和数模转换电路的原理和使用方法
	\item 掌握MCS-51系列单片机中定时器和计数器的使用方法
	\item 掌握使用示波器,信号源对单片机系统进行调试的方法
\end{itemize}
\vspace{3mm}

%%%%%%%%%%%%%%%%%%%%%%%%%%%%%%
%%%%%%%%%%%%%%%%%%%%%%%%%%%%%%
\section{实验原理}
\subsection{数模转换器}
\subsubsection{控制}
单片机系统中,有两个12位DAC和两个比较器,通过特殊功能寄存器DAC0CN实现控制,详见\ref{table_DAC0CN}
\begin{table}[!htbp]
  \centering
  \caption{DAC0CN寄存器结构}
  \label{table_DAC0CN}
\begin{tabular}{|c|c|c|c|c|c|c|c|}
  \hline
  D7&D6&D5&D4&D3&D2&D1&D0\\
    \hline
  DAC0EN&&&DAC0MD1&DAC0MD0&DAC0DF2&DAC0DF1&DAC0DF0\\
  \hline
\end{tabular}
\end{table}
\begin{itemize}
  \item DAC0EN:使能设置,0禁止,1允许
  \item DAC0MD1-0:工作模式,00写入触发,01,10,11分别对应计数器3,4,2
  \item DAC0DF2-0:数据格式,000:右对齐,001右对齐左移一位..1xx左对齐
\end{itemize}
\subsubsection{参考电压}
单片机系统中,使用REF0CN寄存器进行参考电压控制,详见\ref{table_REF0CN}
\begin{table}[!htbp]
  \caption{REF0CN寄存器结构}
  \label{table_REF0CN}
  \begin{tabular}{|c|c|c|c|c|c|c|c|}
    \hline
    D7&D6&D5&D4&D3&D2&D1&D0\\
      \hline
    &&&AD0VRS&AD1VRS&TEMPE&BIASE&REFBE\\
    \hline
  \end{tabular}
\end{table}
\begin{itemize}
  \item AD0VRS:ADC0的参考电压,0为VREF0,1为DAC0输出
  \item AD1VRS:ADC1的参考电压,0为VREF1,1为AV+
  \item TEMPE:内部温度传感器,0禁止,1允许
  \item BIASE:偏置电压允许位,必须设为1
  \item REFBE:内部参考电压允许位,0禁止,1允许
\end{itemize}
%%%%%%%%%%%%%%%%%%%%%%%%%%%%%%%%%%
\subsection{模数转换器}
单片机内部有一个片内ADC0,一个9通道输入多路选择开关和可编程增益放大器,提供100ksps下的真12位进度。
\subsubsection{多路选择器}
使用AMX0SL寄存器选择输入情况,0-7对应AIN0-AIN7,其余为温度传感器
\subsubsection{时钟设置}
ADC0使用ADC0CF寄存器设置SAR的时钟,详见\ref{table_ADC0CF}
\begin{table}[!htbp]
  \caption{ADC0CF寄存器结构}
  \label{table_ADC0CF}
  \begin{tabular}{|c|c|c|c|c|c|c|c|}
    \hline
    D7&D6&D5&D4&D3&D2&D1&D0\\
      \hline
    ADC0SC4&ADC0SC3&ADC0SC2&ADC0SC1&ADC0SC0&AMP0GN2&AMP0GN1&AMP0GN0\\
    \hline
  \end{tabular}
\end{table}
\begin{itemize}
  \item ADC0SC:设置时钟频率为 $\frac{SYSCLK}{CLK_{SAR0}}-1$
  \item AMP0GN:设置内部电压增益,000,001,010,011为1,2,4,8,10x为16,11x为0.5
\end{itemize}
%%%%%%%%%%%%%%%%%%%%%%%%%%%%%%
\subsection{定时器2}
定时器2共有三种工作方式:带捕获的16位定时器/计数器模式,带自装载的16位定时器/计数器,串口0波特率发生器
\begin{table}[!htbp]
  \centering
  \caption{T2CON寄存器结构}
  \begin{tabular}{|c|c|c|c|c|c|c|c|}
    \hline
    D7&D6&D5&D4&D3&D2&D1&D0\\
      \hline
    TF2&EXF2&RCLK0&TCLK0&EXEN2&TR2&C/T2&CP/RL2\\
    \hline
  \end{tabular}

\end{table}
%%%%%%%%%%%%%%%%%%%%%%%%%%%%%%

\subsection{定时器3}
定时器3仅可工作在自装载模式下
\begin{table}[!htbp]
  \centering
  \caption{TMR3CN寄存器结构}
  \begin{tabular}{|c|c|c|c|c|c|c|c|}
    \hline
    D7&D6&D5&D4&D3&D2&D1&D0\\
      \hline
    TF3&&&&&TR3&T3M&T3XCLK\\
    \hline
  \end{tabular}

\end{table}
%%%%%%%%%%%%%%%%%%%%%%%%%%%%%%

\subsection{定时器4}
定时器4和定时器2相同
\begin{table}[!htbp]
  \centering
  \caption{TMR3CN寄存器结构}
  \begin{tabular}{|c|c|c|c|c|c|c|c|}
    \hline
    D7&D6&D5&D4&D3&D2&D1&D0\\
      \hline
    TF3&&&&&TR3&T3M&T3XCLK\\
    \hline
  \end{tabular}

\end{table}
%%%%%%%%%%%%%%%%%%%%%%%%%%%%%%
\section{实验内容}

\subsection{模数数模联调}
\subsubsection{转换输出}
模数转换从信号源输入,将转换结果再通过数模转换输出,输出结果用示波器查看。
\subsubsection{音频回放}
音谱输入已经接到AIN1上(使用AMX0SL选择),音谱输出用DAC1驱动(低位0xD5,高位0xD6)。

\end{spacing}
\end{document}
